\label{form_first}

%%% Formulaire / Form
%%% Remplacer les paramètres des \newcommand par les informations demandées / Replace \newcommand parameters by asked informations
%%%




\newcommand{\NNT}{XXXXXXXXXXXX} 															%% Numéro National de Thèse (donnée par la bibliothèque à la suite du 1er dépôt)/ National Thesis Number (given by the Library after the first deposit)

% \newcommand{\ecodoctitle}{Sciences de l'environnement d'Ile-de-France} 													%% Nom de l'ED. Voir site de l'Université Paris-Saclay / Full name of Doctoral School. See Université Paris-Saclay website
\newcommand{\ecodocacro}{SEIF}																%% Sigle de l'ED. Voir site de l'Université Paris-Saclay / Acronym of the Doctoral School. See Université Paris-Saclay website
\newcommand{\ecodocnum}{129} 																%% Numéro de l'école doctorale / Doctoral School number
\newcommand{\PhDspeciality}{M2 Data and Communication Engineering} 										%% Spécialité de doctorat / Speciality 
\newcommand{\PhDworkingplace}{CentraleSupélec} 										%% Établissement de préparation / PhD working place : l'Université Paris-Sud, l'Université de Versailles-Saint-Quentin-en-Yvelines, l'Université d'Evry-Val-d'Essonne, l'Institut des sciences et industries du vivant et de l'environnement (AgroParisTech), CentraleSupélec,l'Ecole normale supérieure de Cachan, l'Ecole Polytechnique, l'Ecole nationale supérieure de techniques avancées, l'Ecole nationale de la statistique et de l’administration économique, HEC Paris, l'Institut d'optique théorique et appliquée, Télécom ParisTech, Télécom SudParis   
\newcommand{\defenseplace}{Paris-Saclay} 											%% Ville de soutenance / Place of defense
\newcommand{\defensedate}{Date} 															%% Date de soutenance / Date of defense

%%% Établissement / Institution
%%% Si la thèse a été produite dans le cadre d'une co-tutelle, commenter la partie "Pas de co-tutelle" et décommenter la partie "Co-tutelle" / If the thesis has been prepared in guardianship, comment the part "Pas de co-tutelle" and uncomment the part "Co-tutelle"

%%%%%%%%%%%%%%%%%%%%%%%%%
%%% Pas de co-tutelle %%%
%%%%%%%%%%%%%%%%%%%%%%%%%

\newcommand{\logoEtt}{blank}																%% NE PAS MODIFIER / DO NOT MODIFY
\newcommand{\vpostt}{0.1} 																	%% NE PAS MODIFIER / DO NOT MODIFY
\newcommand{\hpostt}{6}																		%% NE PAS MODIFIER / DO NOT MODIFY
\newcommand{\logoEt}{UVSQ} 																	%% Logo de l'établissement de soutenance. Indiquer le sigle / Institution logo. Indicate the acronym : AGRO, CENTSUP, ENS, ENSAE, ENSTA, HEC, IOGS, TPT, TSP, UEVE, UPSUD, UVSQ, X 
\newcommand{\vpos}{0.1}																		%% À modifier au besoin pour aligner le logo verticalement / If needed, modify to align logo vertilcally
\newcommand{\hpos}{11}																		%% À modifier au besoin pour aligner le logo horizontalement / If needed, modify to align logo horizontaly

%%%%%%%%%%%%%%%%%%
%%% Co-tutelle %%%
%%%%%%%%%%%%%%%%%%

%\newcommand{\logoEt}{etab} 																%% Logo de l'université partenaire. Placer le fichier .png dans le répertoire '/media/etab' et indiquer le nom du fichier sans l'extension / Logo of partner university. Place the .png file in the directory '/media/etab' and point the file name without the extension
%\newcommand{\vpos}{0.1}																	%% À modifier au besoin pour aligner les logos verticalement / If needed, modify to align logos vertilcally
%\newcommand{\hpos}{11}																		%% À modifier au besoin pour aligner les logos horizontalement / If needed, modify to align logos horizontaly
%\newcommand{\logoEtt}{etab2}  																%% Logo de l'établissement de soutenance. Le nom du fichier correspond au sigle de l'établissement /  Institution logo. Filename correspond to institution acronym : AGRO, CENTSUP, ENS, ENSAE, ENSTA, HEC, IOGS, TPT, TSP, UEVE, UPSUD, UVSQ, X 
%\newcommand{\vpostt}{0.1} 																	%% À modifier au besoin pour aligner les logos verticalement / If needed, modify to align logos vertilcally
%\newcommand{\hpostt}{6}																	%% À modifier au besoin pour aligner les logos horizontalement / If needed, modify to align logos horizontaly


%%% JURY

% Lors du premier dépôt de la thèse le nom du président n'est pas connu, le choix du président se fait par les membres du Jury juste avant la soutenance. La précision est apportée sur la couverture lors du second dépôt / Choice of the jury's president is made during the defense. Thus, it must be specified only for the second file deposition in ADUM.
% Tous les membres du juty listés doivent avoir été présents lors de la soutenance / All the jury members listed here must have been present during the defense.

%%% Membre n°1 (Président) / Member n°1 (President)
\newcommand{\jurynameA}{Arnaud BOURNEL}
\newcommand{\juryadressA}{Paris-Saclay University}
\newcommand{\juryroleA}{Chairman}

%%% Membre n°2 (Rapporteur) / Member n°2 (Reviewer)
\newcommand{\jurynameBa}{Pierre DUHAMEL}
\newcommand{\juryadressBa}{L2S, CNRS-CentraleSup\'elec-Paris-Saclay University}
\newcommand{\juryroleBa}{Rapporteur}

%%% Membre n°3 (Rapporteur) / Member n°3 (Reviewer)
\newcommand{\jurynameBb}{Name LASTNAME}
\newcommand{\juryadressBb}{Grade, University/Lab (research unit)}
\newcommand{\juryroleBb}{Examiner}

%%% Membre n°4 (Examinateur) / Member n°4 (Examiner)
\newcommand{\jurynameCa}{Name LASTNAME}
\newcommand{\juryadressCa}{Grade, University/Lab (research unit)}
\newcommand{\juryroleCa}{Examiner}

%%% Membre n°5 (Examinateur) / Member n°5 (Examiner)
\newcommand{\jurynameCb}{NGUYEN Linh Trung}
\newcommand{\juryadressCb}{VNU University of Engineering and Technology}
\newcommand{\juryroleCb}{Examiner}

%%% DIRECTION DE THESE
%%% Membre n°6 (Directeur de thèse) / Member n°5 (Thesis supervisor)
\newcommand{\jurynameE}{Natacha CHETCUTI-OSOROVITZ}
\newcommand{\juryadressE}{IDHES, ENS Paris Saclay}
\newcommand{\juryroleE}{Supervisor}

%%% Membre n°7 (Co-directeur de thèse) / Member n°6 (Thesis co-supervisor)
\newcommand{\jurynameF}{Salah EL AYOUBI}
\newcommand{\juryadressF}{L2S, CNRS-CentraleSupélec-Paris-Saclay University}
\newcommand{\juryroleF}{Co-supervisor}

%%% Membre n°8 (Co-directeur de thèse) / Member n°6 (Thesis co-supervisor)
\newcommand{\jurynameG}{TRAN Trong Hieu}
\newcommand{\juryadressG}{VNU University of Engineering and Technology}
\newcommand{\juryroleG}{Co-supervisor}

%% Il est possible d'ajouter des membres supplémentaires selon le même modèle / More jury members can be added according to the same model

\label{layout_first}
%%% Mise en page / Page layout      
%%% NE RIEN MODIFIER EXCEPTÉ LA PARTIE CONCERNANT LE JURY (voir \label{jury}) SI BESOIN / DO NOT MODIFY EXCEPT SECTION CONCERNING JURY (see \label{jury}) IF NEEDED
%%%%%%%%%%%%%%%%%%%%%%%%%%%%%%%%%%%%%%%%%%%%%%%%%%%%%%%%%%%%%%%%%%%%%%%%%%%%%%%%%%%%%%%%%%%%%%%%%%%%%%%%%%%%%%%%%%%%%%%%%%%%%%%%%%%%%%%%%%%%%%%%%%%%%%%%%%%%%%%%%%%%%%%
%%%%%%%%%%%%%%%%%%%%%%%%%%%%%%%%%%%%%%%%%%%%%%%%%%%%%%%%%%%%%%%%%%%%%%%%%%%%%%%%%%%%%%%%%%%%%%%%%%%%%%%%%%%%%%%%%%%%%%%%%%%%%%%%%%%%%%%%%%%%%%%%%%%%%%%%%%%%%%%%%%%%%%%




\thispagestyle{empty}

%    \newgeometry{left=6cm,bottom=2cm, top=1cm, right=1cm}
%	\tikz[remember picture,overlay] \node[opacity=1,inner sep=0pt] at (-13mm,-135mm){\includegraphics{media/bande2.pdf}};

\begin{textblock}{1}(-0.4,0)
	\textblockcolour{bordeau}
	\includegraphics [scale=1]{frontmatter/media/bande_master_eng}
	%\vspace{300mm}
\end{textblock}

%\begin{textblock}{1}(0.6,9.5)
	%\Huge{\rotatebox{90}{\color{white}{\fontsize{38}{54}\selectfont Thèse de doctorat}}}
%\end{textblock}

% \begin{textblock}{1}(1.8,12)
% 	\Large{\rotatebox{90}{\color{white}{NNT : \NNT}}}
% \end{textblock}

% \begin{textblock}{1}(3.2,2)
% 	\textblockcolour{white}
% 	\includegraphics[scale=1]{media/etab/CENTSUP.png}	
% \end{textblock}

\begin{textblock}{1}(10,0.4)
	\textblockcolour{white}
	\includegraphics[scale=0.3]{frontmatter/media/vnu.png}	
\end{textblock}

\begin{textblock}{1}(8.4,0.65)
	\textblockcolour{white}
	\includegraphics[scale=0.035]{frontmatter/media/uet.png}	
\end{textblock}


%% Texte
\begin{singlespace}
\begin{textblock}{12}(3.7,3.5)
	\textblockcolour{white}
	
	\color{Prune}
	\begin{flushright}
        \color{Prune}
		\fontfamily{cmss}\fontseries{m}\fontsize{22}{26}\selectfont
		\Huge{\PhDTitle} \bigskip \\%% Titre de la thèse
		\normalsize
		\color{black}
		% \Large{\textit{\PhDTitleTrad} } \bigskip %% Titre de la thèse traduit
		% \vfill
		\vspace{0.8cm}
		\color{black} %% Couleur noire du reste du texte
		
		\fontfamily{fvs}\fontseries{m}\fontsize{8}{12}\selectfont
		\normalsize \textbf{Master thesis of Paris-Saclay University} \\ \bigskip
		\vspace{5mm}
		\vfill
		% \small{École doctorale n$^{\circ}$\ecodocnum ~\ecodoctitle ~(\ecodocacro)}  \\
		\small{Specialization: \PhDspeciality} \bigskip \\%% Spécialité 
		\small{Research unit: \PhdResearchUnit} \bigskip \\
		\vspace{5mm}
		\vfill 
		% \footnotesize Thèse préparée à l'\PhDworkingplace, sous la direction de \jurynameE ~ (\juryadressE), le co-encadrement de \jurynameF ~ (\juryadressF).\\
		\vspace{5mm}
		\textbf{Thesis presented at \defenseplace, on \defensedate} \bigskip
		\vfill
		\Large{\color{Prune}\textbf{\textsc{\PhDname}}} %% Nom du docteur
		\vfill
	\end{flushright}
	
	\color{black}
	%% Jury
	\begin{flushleft}
		\large{\textbf{Committee}}
	\end{flushleft}
	%% Members of the jury

	\small
	%\begin{center}
	\newcolumntype{L}[1]{>{\raggedright\let\newline\\\arraybackslash\hspace{0pt}}m{#1}}
	\newcolumntype{R}[1]{>{\raggedleft\let\newline\\\arraybackslash\hspace{0pt}}lm{#1}}
	
	\label{jury} 																				%% Mettre à jour si des membres ont été ajoutés ou retirés / Update if members have been added or removed
	\begin{flushleft}
	\begin{tabular}{|p{7cm}l}%{@{} L{9.5cm} R{4.5cm}}
	    \arrayrulecolor{Prune}
		\jurynameA  \\ \juryadressA & \juryroleA \\[5pt]
		\jurynameBa  \\ \juryadressBa & \juryroleBa \\[5pt]
		\jurynameBb  \\ \juryadressBb & \juryroleBb \\[5pt]
		\jurynameCa  \\ \juryadressCa & \juryroleCa \\[5pt]
		\jurynameCb  \\ \juryadressCb & \juryroleCb \\[5pt]
	\end{tabular} 
	\end{flushleft}
	
		%% Jury
	\begin{flushleft}
		\large{\textbf{Thesis Supervision}}
	\end{flushleft}
	%% Members of the jury

	\small
	%\begin{center}
	% \newcolumntype{L}[1]{>{\raggedright\let\newline\\\arraybackslash\hspace{0pt}}m{#1}}
	% \newcolumntype{R}[1]{>{\raggedleft\let\newline\\\arraybackslash\hspace{0pt}}lm{#1}}
	
	\label{directionthese} 																				%% Mettre à jour si des membres ont été ajoutés ou retirés / Update if members have been added or removed
	\begin{flushleft}
	\begin{tabular}{|p{7cm}l}%{@{} L{9.5cm} R{4.5cm}}
	    \arrayrulecolor{Prune}
		\jurynameE  \\ \juryadressE & \juryroleE \\[5pt]
		\jurynameF  \\ \juryadressF & \juryroleF \\[5pt]
        \jurynameG  \\ \juryadressG & \juryroleG \\[5pt]
	\end{tabular} 
	\end{flushleft}
	%\end{center}
\end{textblock}
\end{singlespace}
\afterpage{\blankpage}

\begin{itemize}
	\item The creatioon of Lender-Criteria bipartite graphs
	\item Explain why we could not project Lender-Criteria graphs onto Lender to create Lender-Lender graph
	\item That's why we find communify directly in bipartite graphs
	\item Methodology: better-than-random testing
\end{itemize}

\section{Hypothesis testing on syntehsis data}

\begin{itemize}
	\item How to build a random graph, with community pre-defined
	\item Detecting community in random graph
	\item Quality assessment of the detected community, presented by a numer $Q_{synthesis}$
	\item $Q_{synthesis}$ gives an idea about the best we could get from a data, but depends on the laten parameters when building datasets $p_{in}$, $p_{out}$, $p_{change}$ - the change from community to another
\end{itemize}

\section{Hypothesis testing on real data}

\begin{itemize}
	\item Filter data to keep Lenderes who is active invest from 2019-2023
	\item Repeat the works on random graph for each year data
	\item Then find quality assessment $Q_{real}$ again and compare with $Q_{synthesis}$.
	      Decide if there is a clear community for different criteria (Tags, Sectors, Country)
\end{itemize}

As the begin of this chapter, we are revisiting our hypothesis and testing them out.
Recall that we would like to find community of Lenders, based on their interests in different criteria.
Through the data explornation, we will use 3 different criteria to find if the community exists.
The criteria are: Tags, Sectors and Country.

The first steps is to associate each Lender with Criteria.
In the Kiva data, Lender and Criteria are connected through the Project.
A Lender is considered to be interested in a Criteria if he/she has invested in a Project that has that Criteria.
Note that the relationship between Lender and Project is many-to-many.
The relationship between Project and Criteria could be many-to-many in the case of Tags,
or can be one-to-many in the case of Sectors and Country.
Because of the many-to-many relationship, we will model the data in Graph data type.
This decision is also supported by the fact that the "community finding" is a graph problem.
And by modeling the data in Graph, we can apply the work to other crowdfunding platforms.

\section{Lender-Criteria bipartite graph}

\todo{degree distribution \cite{latapy2006}}

\section{Hypothesis testing on synthesis data}
\section{Hypothesis testing on real data}
\begin{comment}
Content should be contain in this chapter

\begin{itemize}
	\item The dynamic of crowdfunding is rather complex and not well-understood
	\item We would like to understand the role of community in crowdfunding
	      \begin{itemize}
		      \item Hypothesis: there might be communities (explicit or implicit) of lenders (contributors) who willing to contribute to same type of projects again and again
		            \begin{itemize}
			            \item Describle the hypothesis in more details. What is type of projects?
		            \end{itemize}
	      \end{itemize}
	\item The objective of the thesis is to Detect Community Dynamic. If it exists on platforms?
	\item We will test the hypothesis on kiva.org dataset
	\item \textit{Briefly} describe the kiva platform
	      \begin{itemize}
		      \item What is kiva.org and how it works?
		      \item Overview of the data schema: Lender, Project, Sector, Tags, and how those can be viewd as a graph
		      \item What is type of projects in this dataset?
		            \begin{itemize}
			            \item Type means \textit{type of impact} or \textit{localization}
			                  \begin{itemize}
				                  \item Type of impact: impact more on sociality, impact more on environements,... Represented by Sector or Tags
				                  \item Localization: represent by country where borrower live in
			                  \end{itemize}
		            \end{itemize}
	      \end{itemize}
	\item \textit{Briefly} describe how we will work:
	      \begin{itemize}
		      \item Given the graph-like database, we will find community of lenders in the database using \textit{community finding on graphs} techniques
		      \item This is a unsupervised problem
	      \end{itemize}
\end{itemize}
\end{comment}

Crowdfunding has become an increasingly popular way for entrepreneurs and innovators to raise funds for their projects.
It is a process of raising small amounts of money from a large number of people, typically via the Internet.
Researchers have studied crowdfunding from different perspectives, such as the impact of crowdfunding on the economy, or the role of social networks.
The success of crowdfunding campaigns, however, is rather complex and still not well-understood.
In this thesis, we focus on the role of community in crowdfunding.

We hypothesize that there might be communities (explicit or implicit) of lenders (contributors) who are willing to contribute to the same type of projects again and again.
The objective of this thesis is to detect the existing of such communities on crowdfunding platforms.

We will test this hypothesis on the Kiva.org platform,
which is a crowdfunding platform that allows people to lend money to low-income entrepreneurs and students in over 80 countries.
Through API, we will collect data about Lenders, Projects, Sectors, Tags, Countries and the relationships between them.
We then build a graph-like database from the collected data,
and apply community detection on graphs techniques to find communities of lenders in the database.

This is a unsupervised problem, because there is no predefined ground truth about the communities.
Yet it even challenger because the sheer size of the data and the lack of official documentation about the data schema.
We will therefore do heavily data explornation and preprocessing,
before employ better-than-random methodology for find explainable communities.

This master thesis is conducted under the French national research program (ANR) UMICROWD (Understanding, Modeling and Improving the outcome of Crowdfunding campaigns).
UMICROWD is a multidisciplinary initiative that brings together experts in economics, sociology, mathematics, and data science.
The goal of UMICROWD is to study the dynamics of crowdfunding and promote sustainable and socially responsible project funding.

In conclusion, this thesis will explore the complex dynamics of crowdfunding, with a focus on the role of community.
By using the Kiva.org dataset and graph-based community detection techniques,
we aim to shed light on the patterns of lending behavior and the influence of project type on these patterns.
This research could provide valuable data and insights for further research on crowdfunding.

The structure of this thesis is as follows.
In Chapter 2, we review the literature on crowdfunding and community detection on graphs.
In Chapter 3, we describe the Kiva.org platform and the dataset,
as well as understanding basic properties of the dataset.
In Chapter 4, we describe the methodology for community detection on graphs.
First we construct a synthesis dataset to test communtity detection algorithms.
Then we apply the algorithms on the Kiva.org dataset.
In Chapter 5, we discuss the results and propose future works.

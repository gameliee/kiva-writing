\section{Review on Crowdfunding problem}

The Cambridge Dictionary \cite{2023} defines crowdfunding as
"the practice of getting a large number of people to each give small amounts of money in order to provide the finance for a project, typically using the internet".
In this part, we will review some of the most important concepts.
Because of the increasing popularity of crowdfunding, literatures on crowdfunding has been developed in recent years.


Crowdfunding refers to the funding approach that entrepreneurial individuals and groups
fund their ventures or projects by drawing on relatively small contributions
from a large number of individuals through the Internet but not financial intermediaries \cite{mollick2014}.
There are two broad class of \acrshort{cfp}: (i) investment-based and (ii) reward-based \cite{belleflamme2015}.
The first class can be further devide into equity-based, royalty-based and lending-based.
In this type of \acrshort{cf}, funders receive financial returns from the project.
In the second class, funders receive non-monetory compensation.
They support the project because they want to obtain the product the project will produce,
or they believe in the goal and purpose of the project.

The significance of \acrshort{cf} is briefly listed in \cite{xie2019}.
They point out that \acrshort{cf} can support the development of SMEs,
renewable and sustainable enegery project.
Through literature review, they claim that \acrshort{cf} is a hightly significant issue.

In term of success factors of \acrshort{cf}, there are several works.
\cite{colombo2015} observes that \acrshort{cf} has reinforcement characteristic in the sense that contributors generate additional contributors.
\cite{belleflamme2015} shows that entrepreneurs selecting between reward-based and equity-based \acrshort{cf} depends
essentially on the amount of required capital and that equity-based \acrshort{cf} is more suitable for large projects.
Soft information extracted from the descriptive text contributes greatly to \acrshort{cf} success \cite{jiang2020}.
In \cite{xie2019}, authors using a large sample $(N=5128)$ projects collected from Taobao,
the results reveal that the dynamics of \acrshort{cf} market are rather complex.
To the best of our knowledge, \cite{salahaldin2019} is the first work to use a theoretical model to address
the dynamics of a funder’s choice that depend on the remaining project duration
and the first that produces theoretical results that fit empirical observations.
\cite{lindasalahaldin2022} ultilize game theory to show that there exists pivotal moments in CF campaigns.
In that moments, if funders donate more money to the campagins, then the campagins will much likely to be successful.
Otherwise, the campaigns will much likely to be failed.
The authors also shows that the pivotal moments are different for each campaigns.

In conclusion, key concept and categories of \acrshort{cf} are well-indentified in the literature.
Key factors that contribute to the success of \acrshort{cf} are also studied through quanlitave methods.
However, the dynamic of \acrshort{cf} is rather complex and not well-understood.
While existing studies offers insights into potential dynamics, lacking of empirical validation, particularly on large scale, persists.

We would like to find a community-driven motivation or dynamic of succesful crowdfunding campaigns.
In the next chapter, we will introduce the community detection problem and how it can be applied to the crowdfunding problem.

\subsection{plan}

Expectation: 2 pages is more than enough

\begin{itemize}
	\item Definition of crowdfunding (CF)
	      \begin{itemize}
		      \item Quote dictionary \cite{2023}
		      \item List some popular website
	      \end{itemize}
	\item Some type of crowdfunding, or business model of CF
	      \begin{itemize}
		      \item Two board class: investment-based and reward-based \cite{belleflamme2015}
		      \item A paragraph describe investment-based
		      \item A paragraph describe reward-based
	      \end{itemize}
	\item Type of literatures on CF \cite{xie2019}
	      \begin{itemize}
		      \item A pargraph describe the type of literatures, including
		            \begin{itemize}
			            \item Foundation and Concepts
			            \item Significance of CF
			            \item Risks of CF
			            \item CF success factors:
			                  \cite{lindasalahaldin2022} ultilize game theory to show that there exists pivotal moments in CF campaigns.
			                  In that moments, if funders donate more money to the campagins, then the campagins will much likely to be successful.
			                  Otherwise, the campaigns will much likely to be failed.
			                  The authors also shows that the pivotal moments are different for each campaigns.
			                  \cite{salahaldin2019} To the best of our knowledge, this article is the first to use a theoretical model to address the dynamics of a funder’s choice that depend on the remaining project duration and the first that produces theoretical results that fit empirical observations.
			                  We have proposed a decision framework that highlights the impact of CF project duration.
			            \item Gaps in literatures:
			                  Quote "However, the dynamics of how these factors lead to crowdfunding success are not well articulated.
			                  While it is possible to infer from existing studies about the dynamics, empirical validation is lacking, particularly large-scale ones.
			                  To address the gaps, we develop a model next."
		            \end{itemize}
	      \end{itemize}
\end{itemize}

\cite{belleflamme2015}
\cite{mollick2014}
\cite{salahaldin2019}

Translation to the next section:
Because the dynamic of crowdfunding is rather complex and not well-understood.
Few works on CF dynamics from the approach of empricial data analysis on large-scale dataset.
We would like to find a community-driven motivation or dynamic of succesful crowdfunding campaigns.
In the next chapter, we will introduce the community detection problem and how it can be applied to the crowdfunding problem.

\section{Review on Community Detecting problem}

We will review the community detection problem in this section.
The community detection problem is a well-known problem in network science.
Recall that network or a graph is a set of vertex (or nodes) and a set of edges (or links) connecting these vertex.
According to perhap the most famous literature on community finding \cite{fortunato2010},
community refer to groups of vertexs with high concentrations of edges within vertexs within each group,
and low concentrations between vertexs in different groups.
The community structure is also known as \textit{clusters} or \textit{modules} in the literature.

Scientists found community could be beneficial in real-world applications.
For example, finding Web clients who interest in a similar topic and geologically close to each other
can be used to improve the performace of content delivery network,
but setting up a dedicated server for each community.
Recommendation system can also benefit from community structure, by recommending similar items for users in a same community.
Nodes that are centrally located within their clusters,
meaning they share a significant number of edges with other members of the group,
could serve a crucial role in maintaining control and stability within the group.
On the other hand, nodes that are positioned at the boundaries between different modules
are likely to play a key role in mediation,
guiding the interactions and exchanges between various communities.

There are two character of community detection problem.
First, is the lacking of quantitive definition of community.
No defintion is universally accepted.
In fact, the definition often depends on the specific system under study.
Second and even worst, communities are algorithmically defined.
That is, they are just the final product of the algorithms,
without a pricise \textit{a priori} definition \cite{fortunato2010}.

Community detection is a NP-hard problem \cite{fortunato2010},
hence exact algorithms are not feasible for large networks.
It is common to use \textit{approximation algorithms} to provide approximate solution while maintaining low complexity.
Note that approximate algorithms are often non-deterministic, as they provide different solutions for a same problem.
Other than that, it is not possible to approximate the solution for any constant,
as the goodness of approximation strongly depends on the problem at study.
This characteristic of community detection problem provides a challenge for
testing algorithms, let alone apply it to real-world applications.

In the next part, we will review the community detection problem on unipartite graph,
which is the most common type of graph where every nodes are in a same type.
Then we will review the community detection problem on bipartite graph,
a common real-world type of graph where nodes are divided into two types.

\subsection{Unipartite graphs}

\textbf{Basic Notations}

The modeling of graphs is well-known in the literature.
There are 3 parts in a graph.

\begin{itemize}
	\item a set $V$, representing the vertices (or nodes) of the graph
	\item a \textit{binary relation} $E\subseteq V \times V$ representing the edges (or links) of the graph
	\item (Optional) a \textit{function} $\omega: E \rightarrow \mathbb{R}$ representing the \textit{weight} of each edge
\end{itemize}

Denote such a graph with $G(V, E, \omega)$.
When the graph is unweighted, meaning $\omega$ is a constant function, we can use the notation $G(V, E)$.
The \textit{neighbourhood} of vertex $v$ as $N(v) = \{u \in V | (u, v) \in E\}$.
Each element of $N(v)$ is called a \textit{neighbour} of $v$.
The number of nodes in $N(v)$ is called the \textit{degree} of $v$, denoted as $d(v)$.
In the case of weighted graph, the \textit{strength} of $v$, denoted as $s(v)$
which is the sum of the weights of all edges connected to $v$.

The most basic statistic describing such graph is the size of graph, as known as the number of nodes $n = |V|$.
The number of links is $m = |E|$.
The densitive of the graph, the number of edges devided by the number of possible edges, is $\delta(G) = \frac{m}{n(n-1)}$.
It is the probability that two random nodes are linked together.



\textbf{Community detection problem}

In order to tell how good a community finding algorithm, one should use the \textit{quality function}.
It assign a score number to each partition of a graph.
By comparing the score of different partitions, one can tell which partition is better.
But keep in mind that, the answer for the question which partition is better is not universally
but depends on the specific concept of community adopted.

\textit{modularity}\cite{newman2004} is the most popular quality function in the literature.
The ideal is that, a random graph should not have any community structure.
By comparision the edge density of a subgraph and density of a similar random graph (or \textit{null graph}),
it is possible to reveal the community structure of the graph.
The modularity $Q$ can be defined as

\begin{equation}
	Q = \frac{1}{2m} \sum_{ij} \left(A_{ij} - P_{ij}\right) \delta(C_i, C_j)
\end{equation}

Where $m$ is the total number of edges in the graph,
$A$ is the ajdacency matrix of the graph.
$P_{ij}$ is the probability that there is an edge between node $i$ and $j$ in a random graph.
The function $\delta$ is the Kronecker delta function,
which is 1 if $i$ and $j$ in a same community ($C_i = C_j$), and 0 otherwise.
Note that the sum is iterated over all pairs of nodes.
The choice of null graph is abitrairy, but it is common to use the configuration model \cite{newman2004}.
This model consider two node $i, j$ with the degree $k_i, k_j$.
And the corresponding modularity is

\begin{equation}
	Q = \frac{1}{2m} \sum_{ij} \left(A_{ij} - \frac{k_i k_j}{2m}\right) \delta(C_i, C_j)
\end{equation}

One can group the contribution to the sum from the same clusters together,
to obtain the following form where the sum is iterated over all clusters instead of all pairs of nodes.
$n_c$ is the number of clusters, $l_c$ is the number of edges within cluster $c$ (the number of edges connecting vertices in cluster $c$),
$d_c$ is the sum of degree of all vertices in $c$.


\begin{equation}
	Q = \sum_{c=1}^{n_c} \left[\frac{l_c}{m} - \left(\frac{d_c}{2m}\right)^2 \right]
\end{equation}

There are some important properties of this modularity.
It is always smaller than one, and can be negative.
First, large positive values of modularity indicate good partitions \cite{fortunato2010}.
Then, one should not use modularity to compare the goodness of partitions of two graphs that are very different in size.

In the case of weighted graph, the modularity can be written as

\begin{equation}
	Q_w = \frac{1}{2W} \sum_{ij} \left(A_{ij} - \frac{s_i s_j}{2W}\right) \delta(C_i, C_j)
\end{equation}

or the equavielent form

\begin{equation}
	Q_w = \sum_{c=1}^{n_c} \left[\frac{W_c}{W} - \left(\frac{S_c}{2W}\right)^2 \right]
\end{equation}

Here $W$ is the total weight of the graph, $W_c$ is the total weight of cluster $c$,
$S_c$ is the total strength of vertices in cluster $c$.

\todo{resolution parameter}

\note{if cannot possible to write a good review on resolution parameter,
	just mention it and to solve that the modularity is not a swiss-army knife,
	not a one-fit-all solution,
	but a good starting point for community detection problem}

\subsection{Bipartite graphs}

$G(U, V, E, \omega)$



\subsection{plan}
\begin{itemize}
	\item Definition of community finding on a graph
	      Perhap the most famous literature on community finding is \cite{fortunato2010}
	      But a quote from the paper
	      This problem is very hard and not yet satisfactorily solved, despite the huge effort of a large interdisciplinary community of scientists working on it over the past few years.
	\item In real networks, the usually exist a group of vertex that have high density of connection between them,
	      but lower density of connection with the rest of the network.
	      This group of vertex is called community structure or clustering.
	      The community finding problem is to find these group of vertex.
	\item Community finding is a very important problem in network science.
	      Because vertex in community usually share similar properties, or play a similiar role in the graph

	\item Community finding is a hard problem
	      \cite{fortunato2010} show that it is a NP-hard problem
	\item Community finding on dynamic graph
	      \begin{itemize}
		      \item The static graph community analysis is already contreversial
		            Hence not much work on dynamic graph
	      \end{itemize}
	\item Testing algorithms
	      \begin{itemize}
		      \item *planted l-partition model* \cite{fortunato2010}
		      \item
	      \end{itemize}
	      \begin{itemize}
		      \item Most of the works focus on develop algorithms to find community, benchmark on supervised datasets
	      \end{itemize}
	\item Review community finding on
	      \begin{itemize}
		      \item unipartite graph (most of literatures): annotations, algorithms, metrics
		      \item bipartite graph: annotations, algorithms, metrics, projection methods
		      \item Very rate 3-partite graph: just listing the works
	      \end{itemize}

\end{itemize}

\cite{fortunato2010}
\cite{newman2004}
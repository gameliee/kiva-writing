\section{Review on Crowdfunding problem}

The Cambridge Dictionary \cite{2023} defines crowdfunding as
"the practice of getting a large number of people to each give small amounts of money in order to provide the finance for a project, typically using the internet".
Because of the increasing popularity of crowdfunding, literatures on crowdfunding has been developed in recent years.
In this part, we will review some of the most important concepts. 

Crowdfunding refers to the funding approach that entrepreneurial individuals and groups 
fund their ventures or projects by drawing on relatively small contributions 
from a large number of individuals through the Internet but not financial intermediaries \cite{mollick2014}.

There are two broad class of \acrshort{cfp}:



\subsection{plan}

Expectation: 2 pages is more than enough

\begin{itemize}
	\item Definition of crowdfunding (CF)
	      \begin{itemize}
		      \item Quote dictionary \cite{2023}
		      \item List some popular website
	      \end{itemize}
	\item Some type of crowdfunding, or business model of CF
	      \begin{itemize}
		      \item Two board class: investment-based and reward-based \cite{belleflamme2015}
		      \item A paragraph describe investment-based
		      \item A paragraph describe reward-based
	      \end{itemize}
	\item Type of literatures on CF \cite{xie2019}
	      \begin{itemize}
		      \item A pargraph describe the type of literatures, including
		            \begin{itemize}
			            \item Foundation and Concepts
			            \item Significance of CF
			            \item Risks of CF
			            \item CF success factors:
			                  \cite{lindasalahaldin2022} ultilize game theory to show that there exists pivotal moments in CF campaigns.
			                  In that moments, if funders donate more money to the campagins, then the campagins will much likely to be successful.
			                  Otherwise, the campaigns will much likely to be failed.
			                  The authors also shows that the pivotal moments are different for each campaigns.
			                  \cite{salahaldin2019} To the best of our knowledge, this article is the first to use a theoretical model to address the dynamics of a funder’s choice that depend on the remaining project duration and the first that produces theoretical results that fit empirical observations.
			                  We have proposed a decision framework that highlights the impact of CF project duration.
			            \item Gaps in literatures:
			                  Quote "However, the dynamics of how these factors lead to crowdfunding success are not well articulated.
			                  While it is possible to infer from existing studies about the dynamics, empirical validation is lacking, particularly large-scale ones.
			                  To address the gaps, we develop a model next."
		            \end{itemize}
	      \end{itemize}
\end{itemize}

\cite{belleflamme2015}
\cite{mollick2014}
\cite{salahaldin2019}

Translation to the next section:
Because the dynamic of crowdfunding is rather complex and not well-understood.
Few works on CF dynamics from the approach of empricial data analysis on large-scale dataset.
We would like to find a community-driven motivation or dynamic of succesful crowdfunding campaigns.
In the next chapter, we will introduce the community detection problem and how it can be applied to the crowdfunding problem.

\section{Review on Community Detecting problem}

\begin{itemize}
	\item Definition of community finding on a graph
	      Perhap the most famous literature on community finding is \cite{fortunato2010}
	      But a quote from the paper
	      This problem is very hard and not yet satisfactorily solved, despite the huge effort of a large interdisciplinary community of scientists working on it over the past few years.
	\item In real networks, the usually exist a group of vertex that have high density of connection between them,
	      but lower density of connection with the rest of the network.
	      This group of vertex is called community structure or clustering.
	      The community finding problem is to find these group of vertex.
	\item Community finding is a very important problem in network science.
	      Because vertex in community usually share similar properties, or play a similiar role in the graph

	\item Community finding is a hard problem
	      \cite{fortunato2010} show that it is a NP-hard problem
	\item Community finding on dynamic graph
	      \begin{itemize}
		      \item The static graph community analysis is already contreversial
		            Hence not much work on dynamic graph
	      \end{itemize}
	\item Testing algorithms
	      \begin{itemize}
		      \item *planted l-partition model* \cite{fortunato2010}
		      \item
	      \end{itemize}
	      \begin{itemize}
		      \item Most of the works focus on develop algorithms to find community, benchmark on supervised datasets
	      \end{itemize}
	\item Review community finding on
	      \begin{itemize}
		      \item unipartite graph (most of literatures): annotations, algorithms, metrics
		      \item bipartite graph: annotations, algorithms, metrics, projection methods
		      \item Very rate 3-partite graph: just listing the works
	      \end{itemize}

\end{itemize}

\cite{fortunato2010}
\cite{newman2004}